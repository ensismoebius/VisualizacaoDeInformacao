\documentclass[format=sigconf]{acmart}

\title{Enhancing the EEG signals visualization}

\author{André Furlan}

\affiliation{
	\institution{UNESP - Universidade Estadual Paulista Júlio de Mesquita Filho}
	\city{São José do Rio Preto}
	\state{André Furlan}
	\country{Brazil}
}
\email{ensismoebius@gmail.com}

\begin{document}
	
	\begin{abstract}
		Electroencephalogram (EEG) data has inherent complexity, including a large volume, diverse frequencies (e.g., alpha, beta, theta, sigma), and numerous emitting sources (neurons). Additionally, potential electromagnetic interferences from external sources pose challenges during EEG analysis. This work aims to propose effective visualization techniques for EEG data, drawing insights from recent research on EEG visualization. The goal is to create informative visualizations that aid in EEG research by showing detailed representations of the data without the use of misleading gradients. The visualization will include interactive features for subject and stimulus selection, providing a comprehensive understanding of EEG signals with high-contrast color schemes to enhance perceptual clarity.
	\end{abstract}
	
	\maketitle
	
	\section{Introduction}
		\par Electroencephalogram (EEG) data is widely used in neuroscience research to study brain activity. EEG records electrical signals generated by brain neurons and reflects cognitive processes, making it a valuable tool in various applications, including Brain-Computer Interfaces (BCI) and clinical diagnosis of neurological disorders \cite{8937083}. Analyzing EEG data, however, poses challenges due to its complexity, such as a vast number of emitting sources (neurons) and potential electromagnetic interferences from external sources like cell phones, electric motors, and magnetic fields from wires \cite{8937083}. To facilitate EEG data interpretation and visualization, it is crucial to develop effective and efficient visualization techniques that allow researchers to gain valuable insights from the data. This paper proposes a novel visualization method for EEG data, combining line graphs and energy visualizations with a display of the electrode positions in the 10-20 system \cite{sistema10-20}. The proposed visualization will be interactive, allowing users to select subjects and stimuli and explore specific signal details. The goal is to provide researchers with an informative and comprehensive view of EEG data to support their investigations.
	
	\section{Data Acquisition}
		\par To demonstrate the proposed visualization technique, EEG data will be acquired from an existing dataset. The chosen dataset is the "Open access database of EEG signals recorded during imagined speech" \cite{10.1117/12.2255697}. This dataset contains EEG recordings during the imagined pronunciation of vowels and commands, as well as EEG and audio recordings during the actual pronunciation. The dataset includes recordings from 15 Argentinian volunteers between the ages of 24 and 28, with an equal representation of males and females \cite{10.1117/12.2255697}. Although this dataset serves as the initial data source for the proposed visualization, our future research intends to develop a custom EEG database in collaboration with local health institutions.
	
	\section{Database Structure}
		\par Each row in the dataset corresponds to a subject-stimuli-modality set, and the data is arranged as follows: EEG signals from different electrodes, modalities (imagined or pronounced), and stimulus codes representing different categories of stimuli (e.g., letters and commands). The modalities distinguish between imagined and pronounced speech, while the stimulus codes identify specific categories of stimuli. Figure \ref{fig:visu04} shows a sample of the initial rows in the database, and Figure \ref{fig:visu05} displays the last rows.
		
		\begin{figure}[h]
			\centering
			\includegraphics[width=\linewidth]{../presentation/images/visu04}
			\caption{Sample of initial rows in the EEG database}
			\label{fig:visu04}
		\end{figure}
		
		\begin{figure}[h]
			\centering
			\includegraphics[width=.9\linewidth]{../presentation/images/visu05}
			\caption{Sample of last rows in the EEG database}
			\label{fig:visu05}
		\end{figure}
	
	\section{Pre-processing}
		\par To facilitate data manipulation and visualization, pre-processing techniques are applied to the EEG data. The data is grouped and organized to ensure seamless interactions and efficient representation. Figure \ref{fig:visu06} demonstrates the pre-processed and grouped data, ready for visualization.
	
		\begin{figure}[h]
			\centering
			\includegraphics[width=\linewidth]{../presentation/images/visu06}
			\caption{Pre-processed and Grouped EEG Data: Note the columns named after its respective source (i.e., the sensors) and the last three ones "modalidade," "Estímulo," and "Artefatos," which are, respectively, the modality, stimuli and artifacts.}
			\label{fig:visu06}
		\end{figure}
	
	\section{Case Studies}
		\par The proposed visualization method will be evaluated through case studies using real EEG data. The case studies will focus on different subjects and stimuli to demonstrate the effectiveness and utility of the proposed visualization in various scenarios.
	
		\subsection{Visualization Inspired by EPviz \cite{currey2023epviz}}
			\par The proposed visualization method draws inspiration from EPviz, which effectively uses line graphs to visualize EEG data \cite{currey2023epviz}. EPviz visualizes EEG signals captured by all electrodes, providing a general overview of the data. However, the visualization lacks detailed views, which may limit the ability to discern specific patterns and features.
		
			\begin{figure}[h]
				\centering
				\includegraphics[width=\linewidth]{../presentation/images/epviz00}
				\caption{EPviz - Line graph visualization}
				\label{fig:epviz00}
			\end{figure}
	
		\subsection{Visualization Inspired by \cite{9098189}}
			\par Another relevant methodology uses topological color maps to indicate signal energy at each sensor, offering insights into brain activity \cite{9098189}. The visualization in Figure \ref{fig:visu01} provides an informative representation of regions with intense brain activity during the corresponding time period. However, the interpolation of measured values may lead to potential misinterpretations of signal intensities according to gradients.
			
			\begin{figure}[h]
				\centering
				\includegraphics[width=\linewidth]{../presentation/images/visu01}
				\caption{Topological visualization}
				\label{fig:visu01}
			\end{figure}
	
		\subsection{Visualization Inspired by \cite{8937083}}
			\par This methodology shown in Figure \ref{fig:visu02} simplifies the visualization to display brain activity at four different moments \cite{8937083}. However, similar to the previous visualization, the gradient issue persists, potentially leading to incorrect interpretations.
			
			\begin{figure}[h]
				\centering
				\includegraphics[width=\linewidth]{../presentation/images/visu02}
				\caption{Topological visualization over time}
				\label{fig:visu02}
			\end{figure}
	
	\section{Proposal}
		\par The proposed visualization method aims to overcome the limitations of existing techniques. It includes line graphs and energy visualizations combined with a non-gradient representation of electrode positions in the 10-20 system. The interactive features allow users to select subjects and stimuli, explore specific signal details, view the signal's origin, and amplify the displayed curves. Additionally, the visualization will include a bar showing the total energy of the displayed signal interval. The use of high-contrast colors will aid in perceptual differentiation. Figures \ref{fig:g3714} and \ref{fig:g3762} illustrate the general and specific views of the proposed visualization.
	
		\begin{figure}[h]
			\centering
			\includegraphics[width=\linewidth]{../presentation/images/g3714}
			\caption{Specific view of the proposed visualization}
			\label{fig:g3714}
		\end{figure}
		
		\begin{figure}[h]
			\centering
			\includegraphics[width=\linewidth]{../presentation/images/g3762}
			\caption{General view of the proposed visualization}
			\label{fig:g3762}
		\end{figure}
	
	\section{Implementation}
		\par The visualization was implemented using the Python3 \cite{Python3} programming language due to its extensive range of data visualization and manipulation libraries. It is important to note that the final result differs slightly from the proposal, however, this does not compromise the quality of the final product, as can be seen in Figure \ref{fig:screenshot01} and \ref{fig:screenshot02}.
		
		\begin{figure}[h]
			\centering
			\includegraphics[width=\linewidth]{images/screenshot01}
			\caption{The application screenshot in specific mode: All signals are represented separately.}
			\label{fig:screenshot01}
		\end{figure}
		
		\begin{figure}[h]
			\centering
			\includegraphics[width=\linewidth]{images/screenshot02}
			\caption{The application screenshot in general mode: All signals are represented in a single plot.}
			\label{fig:screenshot02}
		\end{figure}
	
		\subsection{User interface discussion}
			\par Focusing on the left side of the visualization: Like is shown in Figure \ref{fig:screenshotleft}, the sensors that are being used are represented by circles with vivid and contrasting colors which matches with its respectively plot on the right side, while the unused sensors are represented only by their names. At the top of each circle, there is a bar representing the corresponding signal's energy captured by that sensor. It is noteworthy that each signal may have different energy levels.
		
			\begin{figure}[h]
				\centering
				\includegraphics[width=\linewidth]{images/screenshotLeft}
				\caption{The sensors and its captured energy are represented by the colored circles and their top bars respectively.}
				\label{fig:screenshotleft}
			\end{figure}
		
			\par The top section shown in Figure \ref{fig:screenshottop} is where the data filters reside: The first one "Selecione a modalidade" is a selector of the modality of the speech: Imagined or pronounced. Just setting this value alone does not affect the visualization because it needs the type of stimuli too. At the "Selecione o estímulo" selector is possible to set the type of stimuli which the visualization are needed. Considering that the previous selector is already set, this will trigger the change on the left and right parts of the screen showing the data visualization for the first subject. The subject can be changed using the "Selecione o sujeito" selector. Finally, the "Selecione o artefato" selector filters if data with artifacts are shown or not shown.
			
			\begin{figure}[h]
				\centering
				\includegraphics[width=\linewidth]{images/screenshotTop}
				\caption{The four selectors of the visualization: "Selecione a modalidade", "Selecione o estímulo", "Selecione o artefato" and "Selecione o sujeito"}
				\label{fig:screenshottop}
			\end{figure}
		
			
		
			\par The right section (Figure \ref{fig:screenshotright}) displays the plots of the signals captured by the sensors. Each plot has a color matching to its respective sensor, and to avoid misinterpretations, there are labels on the right. This section changes when new filters are set and when the switch "Visão geral" depicted in Figure \ref{fig:screenshotbotton} is changed. Here, it is possible to zoom in on data, hide plots by clicking on labels, and even save the results to a file.
			
			\begin{figure}[h]
				\centering
				\includegraphics[width=\linewidth]{images/screenshotRight}
				\caption{The right section of the visualization: The signals may be shown separately or in a single plot.}
				\label{fig:screenshotright}
			\end{figure}
			
			
			\begin{figure}[h]
				\centering
				\includegraphics{images/screenshotBotton}
				\caption{This button switches between "General" and "Specific" modes.}
				\label{fig:screenshotbotton}
			\end{figure}
		\subsection{Source code}
		
			\par The code is designed to load EEG data from a MATLAB file, preprocess it. This visualization uses Panel, Plotly, SciPy, Pandas, BeautifulSoup and NumPy pyhton libraries. The code consists of two main classes: \texttt{Dashdata} and \texttt{Dashboard}.
			
			\par The \texttt{Dashdata} class is responsible for loading and processing EEG data. The main tasks of this class are as follows:
			
			\begin{itemize}
				
				\item{\textbf{loading EEG Data}}: The \texttt{Dashdata} class loads EEG data from a MATLAB file using the \texttt{scipy.io.loadmat} function. The EEG data is stored in a MATLAB struct format, and the class uses \texttt{squeeze\_me} and \texttt{mat\_dtype} options to convert the data into a more usable format.
			
				\item{\textbf{data preprocessing}}: After loading the data, the class creates a Pandas DataFrame called \texttt{dataframe} to store the EEG data. The EEG signals for specific sensors are combined into arrays, and the DataFrame is updated accordingly. This step simplifies data handling and allows easier access to EEG signal data for visualization.
				
				\item{\textbf{setting up Properties}}: The class sets up properties to store essential information, such as the relationship between EEG sensors and their corresponding colors. This mapping is useful for visualizing EEG signals with different colors representing different sensors.
				
				\item{\textbf{mapping Modalities, Stimuli, and Artifacts}}: The \texttt{Dashdata} class provides dictionaries to map descriptive names of modalities, stimuli, and artifacts to their respective numeric codes. This mapping facilitates user interaction, as select widgets can be populated with descriptive options.
				
				\item{\textbf{loading brain Visualization}}: The class loads an SVG image of the brain, which can be used for visualizing the locations of EEG sensors on the brain. This SVG image is used in the dashboard to enhance the user experience and provide context to the EEG data.
			\end{itemize}
			
			\par The \texttt{Dashboard} class is designed to create an interactive EEG Data Visualization Dashboard using Panel and Plotly libraries. It utilizes the \texttt{Dashdata} class to access preprocessed EEG and visual elements data.
			
			\begin{itemize}
				\item{\textbf{widgets and UI Layout}} The dashboard creates various widgets for user interaction, such as selects for modalities, stimuli, artifacts, and subjects. Additionally, a switch widget is used to toggle between general and detailed views of the EEG data. The dashboard's layout is arranged using rows and columns to organize the visual elements effectively.
				
				\item{\textbf{event handling}} The class defines callback methods to handle user interactions with the widgets. These methods update the visualization based on the user's selections for modalities, stimuli, artifacts, subjects, and the general view toggle. The EEG data is filtered and plotted accordingly to provide an interactive and dynamic visualization experience.
				
				\item{\textbf{data visualization}} The dashboard uses Plotly to visualize the EEG data. It creates line plots representing EEG signals for different sensors. Depending on the user's selection, the dashboard can display single-panel or multiple-panel plots. The latter provides a more detailed view of the EEG signals.
			\end{itemize}
			
	\section{Future Work}
		\par As the research progresses, the proposed visualization will be further developed to display the hierarchical decomposition of captured waves into their respective components ($\alpha$, $\beta$, $\theta$, $\sigma$). Additionally, the visualization will show the categories and classes to which each signal belongs.
	
	\section{Code Repository}
	\par To access the source code, please visit:\newline \href{https://github.com/ensismoebius/VisualizacaoDeInformacao}{\textbf{https://github.com/ensismoebius/VisualizacaoDeInformacao}}
	
	\section{Conclusion}
		\par This work proposes a novel visualization method for EEG data, incorporating line graphs and energy visualizations with a non-gradient representation of electrode positions in the 10-20 system. The proposed visualization aims to facilitate EEG data analysis and interpretation by offering interactive features for subject and stimulus selection, as well as specific signal exploration. High-contrast colors will be employed to enhance perceptual clarity. Future research will extend the proposed visualization to include hierarchical representations of signal components and their associated categories and classes.
	
	\begin{acks}
		The author would like to thank the Universidade Estadual Paulista Júlio de Mesquita Filho for its support and resources during this research.
	\end{acks}
	
	\section{References}
	\bibliographystyle{ACM-Reference-Format}
	\bibliography{../presentation/bibliography.bib}
	
\end{document}
