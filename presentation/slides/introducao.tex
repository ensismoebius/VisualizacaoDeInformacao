\begin{frame}
	\frametitle{Introdução}
	\par Dados de Eletroencefalograma (EEG) tem, em sua natureza, um grande volume, variadas frequências que o compõe $\alpha$(8-13Hz), $\beta$(14-30Hz), $\theta$(4-7Hz), $\sigma$(0.5-3Hz) \cite{8937083}, um número extremamente alto de fontes emissoras (os neurônios) assim como múltiplas possíveis interferências eletromagnéticas (telefones celulares, motores elétricos, campos magnéticos provindos de fios, etc.) Esse trabalho se propõe, baseado nas mais recentes pesquisas que incluem visualização de EEG, propor formas efetivas de mostrar, detalhar e sintetizar essas informações de forma que as imagens geradas sejam de alguma ajuda na pesquisa.
\end{frame}
